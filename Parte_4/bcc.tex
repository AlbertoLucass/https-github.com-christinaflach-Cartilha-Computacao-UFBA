\section{Bacharelado em Ciência da Computação} \index{BCC}

	\subsection{Introdução}
	O curso de Computação da UFBA foi o primeiro curso de graduação no Brasil nesta área ao lado do curso de Ciência da Computação da UNICAMP. Começou com o nome de Bacharelado em Processamento de Dados e inciou suas atividades em 03/03/1969 e alia o estudo da arte da ciência e da tecnologia da computação.

	\subsection{Coordenação do Colegiado}
    \begin{itemize}
        \item  Coordenador: Luciano Rebouças de Oliveira
        \item  Vice-Coordenador: Rubisley Lemes
        \item  Sala: 116 do Instituto de Matemática e Estatística(IME)
        \item  Atendimento: Terças e Quintas 14h às 16h
        \item Telefone: (71) 3283-6271
        \item E-mail: ccc@ufba.br
    \end{itemize}
    
    \subsection{Objetivos}
    O objetivo do curso é fornecer conhecimento e práticas para que o egresso esteja situado no campo da tecnologia e da computação. Ele, em sua finalidade, irá nos fornecer uma formação básica porém sólida da teoria da computação, apresentando conceitos da computação e de áreas da tecnologia que utilizem da computação. Também faz parte do objetivo do curso dar uma formação que torne o egresso, um pesquisador da área da computação ou de áreas afins, e capacitá-lo, para que esteja pronto para o mercado de trabalho, tanto no campo industrial, quanto no acadêmico; e a capacidade necessária para ingressar em programas de pós-graduação na área da computação ou áreas afins.
    
    \subsection{Perfil do Egresso}
    O Perfil do Egresso entende-se como as habilidades e competências que o egresso deverá possuir ao concluir o curso. Os egressos do curso devem estar preparados para atuar no mercado de trabalho propondo soluções adequadas que utilizem o computador bem como ter maturidade e conhecimento para atuar de maneira inovadora, contribuindo com o desenvolvimento tecnológico da área da Computação. Eles devem ter uma base científica que os tornem aptos a se tornarem futuros pesquisadores; tenha a capacidade de atuar de maneira multidisciplinar, aplicando a computação em varias áreas de conhecimento. Faz parte do perfil do egresso do curso, atuar como um engenheiro de software, e ter os fundamentos para trabalhar em equipe, principalmente no projeto de sistemas de sistemas.
    
    
    
    \subsection{Distribuição da Carga Horária}
    Algumas normas devem ser definidos pelo Colegiado do Curso. Para o aluno receber o diploma de Bacharel em Ciência da Computação ele deve cursar ma carga horária total de no mínimo de 3.347 horas distribuídas em média em 8 semestres, cumprindo com os seguintes requisitos; 
    \begin{itemize}
        \item  Cursar as disciplinas obrigatórias do curso que totalizam 2.346 horas; 
        \item  Cursar as disciplinas optativas do curso que totalizam 714 horas; 
        \item  Realizar trabalho de conclusão de curso sob a orientação de um professor com apresentação e defesa de uma monografia, com 187 horas de carga horária dividida em 2 componentes curriculares obrigatórios; 
        \item  Cumprir 100 horas de atividades complementares.
    \end{itemize}



\subsection{Elenco de componentes curriculares}
Os componentes curriculares são divididos quanto à natureza em obrigatórios e optativos, e quanto à modalidade em disciplinas, atividades complementares e trabalho de conclusão de curso.
\subsubsection{Pré-requisitos}
 As disciplinas estão relacionadas entre si no que é chamado de pré-requisito, que pode ser obrigatório ou não. Se uma disciplina A é pré-requisito obrigatório de B, você só pode cursar a disciplina B se já tiver sido aprovado na disciplina A. Se uma disciplina A é pré-requisito recomendado de B, recomenda-se que você curse a disciplina A anteriormente ou paralelamente à disciplina B. A disciplina A é pré-requisito recomendado de B quando: 
\begin{itemize}
    \item Noções de conteúdos da disciplina A são utilizados em B, mas o aluno pode obter essas noções sozinho, sem cursar a disciplina A;
    \item Os estudos da disciplina A podem facilitar o aprendizado da disciplina B, assim como podem auxiliar o aluno a propor soluções mais elaboradas para alguns problemas apresentados na disciplina B e até tornar o estudo dos conteúdos de B mais aprofundados;
    \item Conteúdos do ensino médio importantes para o aprendizado dos conteúdos da disciplina B são revisados na disciplina A. 
\end{itemize}
\subsubsection{Componentes curriculares optativos}
O aluno pode cursar os componentes curriculares optativos através de áreas de concentração, perfis complementares ou escolher uma formação genérica. No caso das áreas de concentração, você se direciona a uma área específica da Ciência da Computação, já os perfis complementares direcionam o aluno a disciplinas que façam uma relação da Informática com outras áreas. Caso você não queira escolher nenhuma dessas duas direções, é possível escolher uma formação genérica, na qual o aluno cursa disciplinas optativas mais variadas.

Você pode encontrar exemplos de áreas de concentração e perfis complementares com suas disciplinas no Projeto Pedagógico para o curso de Bacharelado em Ciência da Computação, das páginas 9 à 15, disponível em

https://wiki.dcc.ufba.br/pub/CCC/EstruturadoCurso/reforma2011.pdf
\subsubsection{Componentes curriculares obrigatórios}
 A figura abaixo é um fluxograma que apresenta o andamento do curso em sua situação ideal, com as disciplinas obrigatórias indicadas por código, nome e carga horária. A setas indicam que uma matéria é pré-requisito obrigatório da outra. Por isso, as disciplinas estão divididas por semestre de forma que fiquem ordenadas obedecendo os pré-requisitos e a carga horária semestral fique equilibrada.

As disciplinas optativas estão divididas em 6 de 68 horas e 6 de 51 horas, mas essa divisão não é obrigatória. Desde que curse pelo menos as 714 horas de matérias optativas, a quantidade de disciplinas e a forma em que essas horas estão divididas é de livre escolha do aluno.

Currículo acadêmico do curso de Bacharelado em Ciência da Computação, aprovado em 09/2011, válido para todos os alunos ingressos a partir de 2012.1.

%\begin{table}[h]
%\centering
\begin{longtable}{K{1.7cm}K{6cm}K{1cm}K{0.9cm}K{1.5cm}}
\toprule
\multicolumn{5}{c}{\textbf{Bacharelado em Ciência da Computação}}\\
\midrule
\multicolumn{5}{c}{\textbf{1$^o$ período}}\\
\midrule
 Código & Nome da disciplina & (T-P) & C.H. & Requisitos\\
\midrule
MATA01 & Geometria Analítica & (4-0) & 68 & \\
MATA02 & Cálculo A & (6-0) & 102 \\
MATA37 & Introdução à Lógica de Programação & (2-2) & 68 & \\
MATA38 & Projeto de Circuitos Lógicos & (4-0) & 68 \\
MATA39 & Seminários em Computação & (3-0) & 51 \\
MATA42 & Matemática Discreta I & (4-0) & 68 \\
 \midrule
\multicolumn{2}{c}{TOTAL} & 25 & 425\\
 \midrule
 \multicolumn{2}{c}{TOTAL ACUMULADO} & 23 & 425\\
 \midrule
 \multicolumn{5}{c}{\textbf{2$^o$ período}}\\
\midrule
 Código & Nome da disciplina & (T-P) & C.H. & Requisitos\\
 \midrule
MATA48 & Arquitetura de Computadores a & (4-0) & 68 & MATA38 \\
MATA40 & Estrutura de Dados e Algoritmos I & (2-2) & 68 & MATA37\\
MATA57 & Laboratório de Programação I & (0-3) & 51 & MATA37\\
MATA97 & Matemática Discreta II & (4-0) & 68 & MATA42\\
MATA95 & Complementos de Cálculo & (6-0) & 102 & MATA01 e MATA02\\
MATA07 & Álgebra Linear A & (4-0) & 68 & MATA01\\

 \midrule
\multicolumn{2}{c}{TOTAL} & 25 & 425\\
 \midrule
 \multicolumn{2}{c}{TOTAL ACUMULADO} & 50 & 850\\
 \midrule
 
 \multicolumn{5}{c}{\textbf{3$^o$ período}}\\
\midrule
 Código & Nome da disciplina & (T-P) & C.H. & Requisitos\\
 \midrule
MATA55 & Programação Orientada a Objetos & () & 68 & MATA40 \\
MATA49 & Programação de Software Básico & () & 68 & MATA48, MATA40 e MATA57\\
MATA50 & Linguagens Formais e Autômatos & () & 68 & MATA42\\
MATA47 & Lógica para Computação  & () & 68 & MATA97\\
MAT236 & Métodos Estatísticos  & () & 68 & MATA95\\
FISA75 & Elementos de Eletromagnetismo e Circuitos Elétricos & () & 102 & MATA95\\

 \midrule
\multicolumn{2}{c}{TOTAL} &  & \\
 \midrule
 \multicolumn{2}{c}{TOTAL ACUMULADO} &  & \\
 \midrule
 
\end{longtable}

\noindent
Legenda
%\vspace{0.1cm}
\begin{small}
\begin{description}
\item[C.H.] Carga horária
\item[T] Carga horária de aula teórica
\item[P] Carga horária de aula prática
\end{description}
\end{small}

