\chapter{Código de Honra}

Nós levamos o código de honra a sério e esperamos que você, estudante do DCC, também o respeite.

O desrespeito ao código de honra, prejudica não somente o aprendizado do estudante, mas também afeta a atmosfera de confiança e crescimento pessoal dentro do DCC.

O propósito deste documento é deixar clara as nossas expectativas na certeza de que reduziremos o número de violações no código de honra.

Espera-se que o estudante seja responsável pela produção e envio do próprio código para análise do avaliador.

 Levar crédito pelo trabalho produzido por outra pessoa, se passando como o autor, é considerado plágio e será punido de acordo com o regime disciplina estabelecido pela UFBA.
 
 \section{3 cláusulas}

De forma individual e coletiva, o estudante do DCC:
  \begin{enumerate}
  \item Não fornecerá ou receberá ajuda em avaliações;
  \item Não fornecerá ou receberá ajuda não autorizada em atividades em laboratório, na elaboração de relatórios, ou em qualquer outra atividade sujeita a pontuação;
  \item Atuará de forma a garantir que o código de honra seja seguido por todos os demais estudantes.
  \end{enumerate}
  
  Os docentes do DCC confiam na integridade acadêmica dos seus estudantes, abstendo-se de tomar precauções desnecessárias para impedir as formas de desonestidade mencionadas acima. O corpo docente, na medida do possível, evitará práticas didáticas que estimulem a violação do código de honra.
  
  Apesar do docente ter o direito e a obrigação de definir os critérios de avaliação, os estudantes e o docente podem reestabelecer esses critérios para obter um melhor rendimento de aprendizado na disciplina.
  
  \section{3 regras}
  
  \begin{enumerate}
\item Não consultar soluções ou códigos de um programa que não é seu.
\item Não compartilhar o código com outros estudantes.
\item Você deve indicar no seu trabalho as ajudas que você recebeu
\end{enumerate}

Considera-se como plágio:
\begin{itemize}
\item Apresentar o trabalho que é copiado ou derivado do trabalho dos outros e apresentado como o seu próprio
\item  Usar parte ou todo de uma solução a partir da Internet ou de uma solução de outro aluno (passado ou presente).
\end{itemize}
  
Consultar o  código de solução de outra pessoa, a fim de determinar como resolver o seu próprio problema também é uma infração do código de honra. 

Evite pedir para o seu colega compartilhar o código contigo e não discuta detalhes minuciosos de implementação a ponto do seu colega acabar escrevendo um código igual ao seu. Qualquer ajuda deve ser limitada e não pode chegar ao ponto de outra pessoa modificar o seu código.

Tudo bem em discutir ideias, desde que isso não viole as regras 1 e 2. Não publique as soluções dos problemas online.  Indique o nome de quem o ajudou e que tipo de ajuda foi recebida e informe a fonte, em caso de ajuda livros, sites, apostilas, etc.


  
