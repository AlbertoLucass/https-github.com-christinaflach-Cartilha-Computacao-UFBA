\section{Sistemas de Informação}
\index{BSI}
\subsection{Introdução}

O curso de Bacharelado em Sistemas de Informação da UFBA foi criado em 2010. Embora ainda seja uma área recente, já conquistou um espaço relevante no mercado de trabalho. O bacharel em Sistemas de Informação deve planejar e organizar o processamento, o armazenamento e a recuperação de informações, de modo que estas possam ser disponibilizadas aos usuários.

\textcolor{red}{Texto repetido}

O Bacharelado em Sistemas de Informação(SI) foi implantado na Universidade Federal da Bahia(UFBA) no ano de 2010. Apesar de ter menos tempo vigente na Universidade que o curso de Bacharelado em Ciência da Computação (criado em 1968), ambos têm afinidade e andam juntos durante o curso.

O curso alia conhecimentos da computação com gestão em geral, proporcionando ao estudante aprendizados importantes sobre os componentes dos Sistemas de Informação.
  
  \subsection{Coordenação do Colegiado}
       \begin{itemize}
           \item Coordenador: Professor Ricardo Araújo Rios
           \item Vice-Coordenador: Professor Paul Regnier
           \item Sala: 115 do Instituto de Matemática(IM)
           \item Atendimento: Segunda-feira, das 18h às 20h30
           \item Telefone: (71) 3283-6267
           \item E-mail: csi@ufba.br
       \end{itemize}
       
\subsection{Estrutura Curricular}
 \paragraph{Carga Horária}
       O currículo do curso possui uma distribuição de horas-aula seguindo o padrão:
       \begin{itemize}
           \item Eixo Central: 85 por cento, distribuídos por:\\
                Disciplinas obrigatórias: 2346 horas\\
                Atividades complementares: 100 horas\\
                Trabalho de Conclusão do Curso(TCC): 187 horas\\
           \item Componentes curriculares optativos: 15 por cento, distribuídos por:\\
                Disciplinas optativas: 544 horas.\\
            \end{itemize}
        Totalizando a carga horária de 3177 horas.
    
 \paragraph{Componentes Curriculares}
     \begin{itemize}
            \item Eixo Central: As matérias e atividades proporcionadas por esta área fornecem grande formação em Sistema de Informação, Computação e Engenharia de Software. É obrigatório para os alunos a conclusão da carga horária do Eixo, pois as disciplinas são consideradas essenciais e feitas como base para a formação de um bacharel em SI.
            \item Componentes curriculares optativos:
        O currículo permite que o estudante opte por disciplinas que o especializem em sistemas WEB ou e Informática em Saúde.
            \item Disciplinas livres:
        São matérias que têm relação com a Computação ou com formação complementar, que não estão organizadas pelas áreas de concentração.
        \end{itemize}
    
\subsection{Habilidades e Competências}
 \paragraph{Ao concluir o curso de SI, pode-se exercer as seguintes funções:}
       \begin{itemize}
           \item Analista, projetista e programador de sistemas de informação;
           \item Gerente de projetos em informática;
           \item Consultoria em TI e Sistemas de Informação;
           \item Analista de suporte a ambientes computacionais;
           \item Gerente de suporte a ambientes computacionais;
           \item Gerente de divisões organizacionais de tecnologia da informação.
       \end{itemize}
       
\subsection{Pós-formação}
 \paragraph{O que se espera de uma pessoa formada em SI?}
    \begin{itemize}
        \item Solucionar problemas computacionais usando as tecnologias atuais;
        \item Trabalhar em equipe;
        \item  Prestar serviços de Consultoria;
        \item Interagir no meio de forma criativa e modificadora;
        \item Possuam formação complementar em Gestão;
        \item Possam atuar no mercado de trabalho em prestação de serviços de consultoria.
        \end{itemize}

%\textcolor{red}{Utilize o sistema de tabelas adotado pelos alunos que montaram as grades de LCC e BCC}