\chapter{Glossário acadêmico}

\begin{description}
\item[Sistema acadêmico] O sistema acadêmico adotado na UFBA é o de créditos, com matrículas em períodos semestrais.
\item[Crédito] Unidade de medida de trabalho escolar que corresponde a 17 (dezessete) horas de aula.
\item[Ano acadêmico] Período do ano no qual são desenvolvidas as atividades escolares. No ensino superior, geralmente fala-se em semestre acadêmico (ou letivo). Na UFBA, cada semestre letivo tem atualmente 17 semanas.
\item[Projeto pedagógico] É o instrumento de concepção do curso. É um projeto no qual devem ser definidos a concepção, estrutura, procedimentos de avaliação e instrumentos de apoio do curso.
\item[Currículo] Agregado de disciplinas que fazem parte de um determinado curso. É o programa completo fornecido por um curso.
\item[Plano de estudo] Consiste no programa curricular e nas diretrizes de ensino de uma determinada disciplina ou curso.
\item[Disciplinas] Programas de aulas que abordam campos específicos de conhecimento, também chamadas de matérias. Os currículos de cursos superiores incluem disciplinas obrigatórias (precisam ser cursadas para que o aluno finalize o curso), eletivas (escolhidas pelo aluno dentre quaisquer disciplinas que figurem na matriz curricular do curso) e optativas (escolhidas pelo aluno, de fora da matriz curricular).
\item[Ementa] Sumário de assuntos abordados/a ser abordados em uma dada disciplina.
\item[Pré-requisito] Algumas disciplinas possuem pré-requisitos, isto é, outras disciplinas que precisam ser completadas antes que se possa cursar a disciplina desejada. 
\item[Regime didático] Conjunto de normas que regulamenta o desenvolvimento e execução das atividades didáticas.
\item[Calendário acadêmico] Calendário de datas relevantes à vida acadêmica. Contém datas de seminários, períodos para realização de procedimentos específicos (como matrículas, trancamentos, solicitações, etc), entre outros.
\item[Carga horária complementar] É a carga horária advinda de atividades externas ao curso. Inclui atividades de pesquisa e extensão, estágios, eventos acadêmicos, etc. Cada curso exige uma quantidade específica de carga horária complementar.
\item[Projeto final de curso] Atividade obrigatória para conclusão do curso. É um projeto conduzido pelo aluno apenas com acompanhamento de um professor, concedendo-lhe a oportunidade de aplicar os conhecimentos adquiridos durante um curso.
\item[Diretório acadêmico] Órgão composto por alunos e representativo do corpo discente de um curso junto à administração da instituição. Existe também o DCE, Diretório Central dos Estudantes, que é a representação central e máxima de todo o corpo discente.
\item[Empresa júnior] Empresa formada apenas por estudantes do curso, sob orientação de um professor, pensada de modo a oferecer experiência de mercado aos alunos ainda na universidade.

\item[Reitor] é a autoridade superior da UFBA e representante legal em todos os atos e feitos judiciais e extra-judiciais.

\item[Gabinete do Reitor] tem a finalidade de prestar assistência ao Reitor na execução de suas atribuições.

\item[Conselhos] são entidades que representam diversas classes na UFBA, como por exemplo: Professores, Estudantes e até membros da comunidade. O Conselho mais ligado aos alunos é o Conselho Universitário.

\item[Representação estudantil] está voltada para a necessidade de jovens construírem sua participação na política estudantil, que contribuirá para sua identificação de necessidades junto aos processos de formação, auxiliando a qualificá-los através de uma participação ativa junto aos segmentos das diversas instâncias da instituição educativa, tendo como meta a formação alicerçada em valores sólidos.

\item[Secretaria Geral dos Cursos] é responsável pelos documentos referentes à vida acadêmica dos estudantes.

\end{description}